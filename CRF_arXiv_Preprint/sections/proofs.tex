\section{Mathematical Proofs}
\label{sec:proofs}

\subsection{Maximum Resonance Theorem}

\begin{theorem}[Maximum Resonance]
\label{thm:max_resonance}
For any delay vector $\vec{\tau} = (\tau_1, \tau_2, \dots, \tau_n) \in \mathbb{R}^n$,
\[
\mathcal{R}(\vec{\tau}) = \exp\left(-\gamma \sum_{1 \leq i < j \leq n} (\tau_i - \tau_j)^2\right) \leq 1
\]
Equality $\mathcal{R}(\vec{\tau}) = 1$ holds if and only if $\tau_1 = \tau_2 = \cdots = \tau_n$.
\end{theorem}

\begin{proof}
Consider the pairwise delay differences:
\begin{enumerate}
\item For all $i < j$, we have $(\tau_i - \tau_j)^2 \geq 0$ since squares are non-negative.
\item Thus the sum $S = \sum_{1 \leq i < j \leq n} (\tau_i - \tau_j)^2 \geq 0$.
\item Multiplying by $-\gamma$ (where $\gamma > 0$):
\[
-\gamma S \leq 0
\]
\item The exponential function is monotonically increasing, so:
\[
\mathcal{R} = e^{-\gamma S} \leq e^0 = 1
\end{enumerate}\]

\textbf{Equality condition}: 
$\mathcal{R} = 1$ occurs iff $-\gamma S = 0$ iff $S = 0$ iff $(\tau_i - \tau_j)^2 = 0$ for all $i,j$ iff $\tau_i = \tau_j$ for all $i,j$. 

Thus perfect resonance requires all delays to be equal.
\end{proof}

\subsection{Gradient of the Resonance Field}

\begin{theorem}[CRF Gradient]
\label{thm:gradient}
The gradient of the resonance field with respect to delay $\tau_k$ is:
\[
\frac{\partial \mathcal{R}}{\partial \tau_k} = -2\gamma \mathcal{R} \sum_{i \neq k} (\tau_k - \tau_i)
\]
\end{theorem}

\begin{proof}
Define the energy function:
\[
E(\vec{\tau}) = -\gamma \sum_{1 \leq i < j \leq n} (\tau_i - \tau_j)^2
\]
so that $\mathcal{R} = e^{E}$. By the chain rule:
\[
\frac{\partial \mathcal{R}}{\partial \tau_k} = \frac{d\mathcal{R}}{dE} \cdot \frac{\partial E}{\partial \tau_k}
\]

\textbf{Step 1}: Derivative of $\mathcal{R}$ with respect to $E$:
\[
\frac{d\mathcal{R}}{dE} = \frac{d}{dE} e^E = e^E = \mathcal{R}
\]

\textbf{Step 2}: Derivative of $E$ with respect to $\tau_k$. Expand the sum:
\[
E = -\gamma \sum_{i<j} (\tau_i - \tau_j)^2
\]
The derivative $\partial E / \partial \tau_k$ depends only on terms containing $\tau_k$. For each pair $(k,i)$ with $i \neq k$:
\[
\frac{\partial}{\partial \tau_k} (\tau_k - \tau_i)^2 = 2(\tau_k - \tau_i)
\]
There are $(n-1)$ such terms. Thus:
\[
\frac{\partial E}{\partial \tau_k} = -\gamma \cdot 2 \sum_{i \neq k} (\tau_k - \tau_i)
\]

\textbf{Combining results}:
\[
\frac{\partial \mathcal{R}}{\partial \tau_k} = \mathcal{R} \cdot \left(-2\gamma \sum_{i \neq k} (\tau_k - \tau_i)\right) = -2\gamma \mathcal{R} \sum_{i \neq k} (\tau_k - \tau_i)
\]
\end{proof}