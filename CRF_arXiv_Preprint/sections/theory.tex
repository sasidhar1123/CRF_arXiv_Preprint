\section{Core CRF Formalism}  
\label{sec:theory}  

\subsection{Delay Vector Space}  
\begin{definition}[Delay Vector]  
In any AI system with $n$ components (neural layers, sensors, etc.), each has a \textit{processing delay} $\tau_i$ (time to complete its task). Together, they form a vector in $n$-dimensional space:  
\[  
\vec{\tau} = (\tau_1, \tau_2, \dots, \tau_n) \in \mathbb{R}^n  
\]  
This is called the \textbf{CRF space}.  
\end{definition}  

\textbf{Intuition}: Imagine a football team where:  
- Each player ($\tau_i$) has their own running speed  
- CRF space is the "coordination map" showing who's falling behind  

\subsection{Resonance Field $\mathcal{R}(\vec{\tau})$}  
\begin{definition}[Resonance Field]  
The "harmony" of the system is measured by:  
\[  
\mathcal{R}(\vec{\tau}) = e^{-\gamma \sum_{1 \leq i < j \leq n} (\tau_i - \tau_j)^2}  
\]  
where $\gamma > 0$ is a sensitivity parameter (default: $0.01$).  
\end{definition}  

\textbf{Key Properties}:  
1. \textit{Range}: $0 < \mathcal{R} \leq 1$  
2. \textit{Perfect Alignment}: $\mathcal{R} = 1$ only when $\tau_1 = \tau_2 = \cdots = \tau_n$  
3. \textit{Physics Analogy}: Like tuning forks vibrating together - matched timing creates loud resonance.  

\begin{theorem}[Maximum Resonance]  
\label{thm:maxR}  
For any $\vec{\tau}$, $\mathcal{R}(\vec{\tau}) \leq 1$. Equality holds iff all delays are equal.  
\end{theorem}  
\begin{proof}  
Since $(\tau_i - \tau_j)^2 \geq 0$, the exponent $-\gamma \sum (\tau_i - \tau_j)^2 \leq 0$. Thus $\mathcal{R} \leq e^0 = 1$. Equality holds when all $\tau_i = \tau_j$.  
\end{proof}  

\subsection{Alignment Force: The CRF Gradient}  
\begin{definition}[Alignment Gradient]  
The force pulling component $k$ into sync is:  
\[  
\frac{\partial \mathcal{R}}{\partial \tau_k} = -2\gamma \mathcal{R} \sum_{i \neq k} (\tau_k - \tau_i)  
\]  
\end{definition}  

\textbf{Mechanism}:  
\begin{enumerate}  
\item If $\tau_k > \tau_i$ (too slow), gradient $<0$ → \textit{decrease} $\tau_k$  
\item If $\tau_k < \tau_i$ (too fast), gradient $>0$ → \textit{increase} $\tau_k$  
\end{enumerate}  

\textbf{Optimization Rule}:  
\[  
\tau_k \leftarrow \tau_k + \eta \frac{\partial \mathcal{R}}{\partial \tau_k}  
\]  
where $\eta$ = learning rate (controls adjustment speed).  

\subsection{Causality Constraint}  
\begin{principle}[No Time Travel]  
Delays must respect cause-effect order:  
\[  
\text{If component } A \text{ feeds data to } B \text{, then } \tau_B > \tau_A  
\]  
Formally, dependencies form a \textbf{Directed Acyclic Graph (DAG)}.  
\end{principle}  

\textbf{Implementation}:  
\begin{algorithmic}  
\State For each update:  
\State $\quad \tau_k^{\text{new}} \gets \tau_k + \eta \frac{\partial \mathcal{R}}{\partial \tau_k}$  
\State $\quad \texttt{assert } \tau_k^{\text{new}} > \max(\text{input delays})$  
\end{algorithmic}  

\subsection{Delay Compression}  % MODIFIED SECTION  
\begin{problem}  
For large $n$, computing $\sum_{i<j} (\tau_i-\tau_j)^2$ is slow ($\sim n^2$ terms).  
\end{problem}  

\begin{solution}[Top-$k$ Value Compression]  
Instead of full compression, keep only the $k$ most \textit{influential} delays:  
\begin{enumerate}  
\item Compute mean delay: $\bar{\tau} = \frac{1}{n}\sum_{i=1}^n \tau_i$  
\item Calculate deviations: $\delta_i = |\tau_i - \bar{\tau}|$  
\item Select indices $I = \{i_1, \dots, i_k\}$ with largest $\delta_i$  
\item Compressed representation:  
\[  
\vec{z} = \begin{pmatrix} \tau_{i_1} \\ \vdots \\ \tau_{i_k} \end{pmatrix} \in \mathbb{R}^k  
\]  
\item Reconstruct full vector:  
\[  
\tau'_j = \begin{cases}  
\tau_j & j \in I \\  
\bar{\tau} & \text{otherwise}  
\end{cases}  
\]  
\end{enumerate}  
Then compute $\mathcal{R}(\vec{\tau}') \approx \mathcal{R}(\vec{\tau})$.  
\end{solution}  

\begin{theorem}[Error Bound]  
\label{thm:error}  
If $\|\vec{\tau} - \vec{\tau}'\| < \delta$, then:  
\[  
|\mathcal{R}(\vec{\tau}) - \mathcal{R}(\vec{\tau}')| < 2\gamma n^2 \delta \max|\tau_i - \tau_j|  
\]  
\end{theorem}  
\begin{proof}  
From Mean Value Theorem, since $\mathcal{R}$ is Lipschitz continuous with constant $2\gamma n^2 \max|\tau_i - \tau_j|$.  
\end{proof}  

\subsection{Visual Summary}  
\begin{figure}[h]  
\centering  
\begin{tikzpicture}[node distance=2cm]  
\node (start) [startstop] {AI System};  
\node (tau) [process, below of=start] {Measure $\vec{\tau}$};  
\node (compress) [process, right of=tau, xshift=2cm] {Top-$k$ Compression};  % UPDATED  
\node (R) [process, below of=tau] {Compute $\mathcal{R}(\vec{z})$};  
\node (grad) [process, below of=R] {Calculate $\frac{\partial \mathcal{R}}{\partial \tau_k}$};  
\node (update) [process, below of=grad] {Update $\tau_k$};  
\node (dag) [decision, right of=update, xshift=2cm] {DAG valid?};  
\node (end) [startstop, below of=update] {Aligned System};  

\draw [arrow] (start) -- (tau);  
\draw [arrow] (tau) -- (R);  
\draw [arrow] (tau) -- (compress);  
\draw [arrow] (compress) |- (R);  
\draw [arrow] (R) -- (grad);  
\draw [arrow] (grad) -- (update);  
\draw [arrow] (update) -- (dag);  
\draw [arrow] (dag) -- node[right] {yes} (end);  
\draw [arrow] (dag) -- node[above] {no} ++(3,0) |- (grad);  
\end{tikzpicture}  
\caption{CRF alignment workflow with top-$k$ compression}  % UPDATED  
\label{fig:workflow}  
\end{figure}  
